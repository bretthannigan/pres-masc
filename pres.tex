\documentclass[10pt,usenames,dvipsnames]{beamer}

\usetheme[progressbar=frametitle]{metropolis}

\definecolor{ubcBlue}{RGB}{12,35,68}
\definecolor{ubcBlue1}{RGB}{0,85,183}
\definecolor{ubcBlue2}{RGB}{0,167,225}
\definecolor{ubcBlue3}{RGB}{64,180,229}
\definecolor{ubcBlue4}{RGB}{110,196,232}
\definecolor{ubcBlue5}{RGB}{151,212,223}

% \setbeamercolor{normal text}{bg=ubcBlue1}
\setbeamercolor{alerted text}{bg=ubcBlue1, fg = ubcBlue}
\setbeamercolor{example text}{fg=ubcBlue1, bg=ubcBlue1}
\setbeamercolor{title separator}{fg = ubcBlue, bg=ubcBlue}
\setbeamercolor{progress bar}{bg=ubcBlue4, fg=ubcBlue1}
\setbeamercolor{progress bar in head/foot}{bg=ubcBlue4, fg=ubcBlue1}
\setbeamercolor{progress bar in section page}{bg=ubcBlue4, fg=ubcBlue1}
\setbeamercolor{frametitle}{bg=ubcBlue}


\usepackage{appendixnumberbeamer}

\usepackage{booktabs}
\usepackage[scale=2]{ccicons}

\usepackage{pgfplots}
\usepgfplotslibrary{dateplot}

\usepackage{xspace}
\newcommand{\themename}{\textbf{\textsc{metropolis}}\xspace}

%% My Packages
% Text
\usepackage{titlecaps}
% Mathematics and Symbols
\usepackage{amssymb}
\usepackage{amsmath}
\usepackage{amsthm}
\usepackage{mathtools}
\usepackage[electronic]{ifsym}
\usepackage{algorithm}
\usepackage{algpseudocode}
% Time and Units
\usepackage{datetime2}
\usepackage{siunitx}
\sisetup{detect-all}
% Tables
\usepackage{tabularx}
\newcolumntype{P}[1]{>{\RaggedRight\arraybackslash}p{#1}} % ragged-right version of "p" column type
\newcolumntype{C}[1]{>{\centering\arraybackslash$}p{#1}<{$}} % For fixed-width array columns.
\usepackage{multicol}
\usepackage{multirow}
\usepackage{diagbox}
\usepackage{threeparttable}
\usepackage{makecell}
% Page Formatting
\usepackage{rotating}
\usepackage{pdflscape}
\usepackage{subcaption}
\usepackage{fancyhdr}
\usepackage{ragged2e}
\usepackage{varwidth}
% Graphics and Drawing
\graphicspath{ {img/} }	
\usepackage{tikz}
\usetikzlibrary{arrows, positioning, dsp, chains, fit, calc, matrix}
\usepackage[american]{circuitikz}
\usepackage{pgfplots}
\pgfplotsset{compat=1.3}
\usepgfplotslibrary{fillbetween}
\pgfdeclarelayer{bg}    % declare background layer
\pgfsetlayers{bg,main}  % set the order of the layers (main is the standard layer)
\usepackage{epstopdf}
\epstopdfsetup{outdir=./}
% Files
\usepackage{filecontents}

\makeatletter
\newsavebox{\mybox}
\setbeamertemplate{frametitle}{%
  \nointerlineskip%
  \savebox{\mybox}{%
      \begin{beamercolorbox}[%
          wd=\paperwidth,%
          sep=0pt,%
          leftskip=\metropolis@frametitle@padding,%
          rightskip=\metropolis@frametitle@padding,%
        ]{frametitle}%
      \metropolis@frametitlestrut@start\insertframetitle\metropolis@frametitlestrut@end%
      \end{beamercolorbox}%
    }
  \begin{beamercolorbox}[%
      wd=\paperwidth,%
      sep=0pt,%
      leftskip=\metropolis@frametitle@padding,%
      rightskip=\metropolis@frametitle@padding,%
    ]{frametitle}%
  \metropolis@frametitlestrut@start\insertframetitle\metropolis@frametitlestrut@end%
  \hfill%
  \raisebox{-\metropolis@frametitle@padding}{\includegraphics[height=\dimexpr\ht\mybox+\metropolis@frametitle@padding\relax]{UBC-Metropolis-Beamer/2_2016_UBCNarrow_Signature_ReverseCMYK}}%
    \hspace*{-\metropolis@frametitle@padding}
  \end{beamercolorbox}%
}
\makeatother

\title{On the Design of Stable, High Performance Sigma Delta Modulators}
\subtitle{M.A.Sc. Thesis Defence}
\date{\DTMdisplaydate{2018}{12}{04}{}}
\author{Brett Hannigan\\ \texttt{bch@alumni.ubc.ca}}
\institute{School of Biomedical Engineering\\ University of British Columbia}
% \titlegraphic{\hfill\includegraphics[height=1.5cm]{logo.pdf}}

\begin{document}

\maketitle

\begin{frame}{Table of Contents}
	\setbeamertemplate{section in toc}[sections numbered]
	%\tableofcontents[hideallsubsections]
	\tableofcontents
\end{frame}

\section{Introduction}

\begin{frame}[fragile]{Project Objectives}

\metroset{block=fill}
\begin{center}
	\includegraphics[width=6cm]{logos.png}
\end{center}
\begin{block}{Primary Objective}
	To develop a systematic method of design for sigma delta A/D converters for the recording of bio-signals.
\end{block}
Ideally, the goals of the method are to:
\begin{itemize}
	\item Model the nonlinear system accurately in a way that allows analysis of existing designs.
	\item Reduce dependence on simulation.
	\item Provide a way to design guaranteed stable sigma delta modulators in a way that minimizes conservatism.
\end{itemize}

\end{frame}

\begin{frame}{Principles of Sigma Delta Modulation}

\begin{figure}
\noindent\makebox[\textwidth]{
	\centering
	\begin{tikzpicture}[ampersand replacement=\&,scale=0.75, every node/.style={scale=0.75}]
		\node[coordinate] (g-y0) at (0,0) {};
		\node[coordinate] (g-y1) at (5,0) {};
		\node[coordinate] (g-y2) at (10,0) {};
		\begin{axis}[
			at={(g-y0)},
			anchor=center,
			width=6cm, height=3.75cm,
			anchor=center, 
			xmin=0, xmax=4,
			ymin=-100, ymax=300,
			axis x line=none, axis y line=none, axis line style={-},
			xticklabels={0,1,2,3,\SI{4}{\second}}, xtick={0,1,2,3,4},
			yticklabels={, 0, \SI{300}{\micro\volt}}, ytick={-100, 0, 300}
			]
			\addplot[solid,black] table [x=t, y=y, col sep=comma] {data/comparison-y0.csv};
		\end{axis}
		
		\begin{axis}[
			at={(g-y1)},
			anchor=center,
			width=6cm, height=3.75cm,
			anchor=center, 
			xmin=0, xmax=4,
			ymin=-100, ymax=300,
			axis x line=none, axis y line=none, axis line style={-},
			xticklabels={0,1,2,3,\SI{4}{\second}}, xtick={0,1,2,3,4},
			yticklabels={, 0, \SI{300}{\micro\volt}}, ytick={-100, 0, 300}
			]
			\addplot[solid,black] table [x=t, y=y, col sep=comma] {data/comparison-y1.csv};
		\end{axis}
		
		\begin{axis}[
			at={(g-y2)},
			anchor=center,
			width=6cm, height=3.75cm,
			anchor=center, 
			xmin=0, xmax=4,
			ymin=-100, ymax=300,
			axis x line=none, axis y line=none, axis line style={-},
			xticklabels={0,1,2,3,\SI{4}{\second}}, xtick={0,1,2,3,4},
			yticklabels={, 0, \SI{300}{\micro\volt}}, ytick={-100, 0, 300}
			]
			\addplot[solid,black] table [x=t, y=y, col sep=comma] {data/comparison-y2.csv};
		\end{axis}
		
	\end{tikzpicture}}
	\caption{An example EEG signal digitized to 5 bits with na\"{i}ve quantization (left), 10 times oversampled quantization (middle), and first-order sigma delta modulation (right).}
\end{figure}

\metroset{block=fill}
\begin{block}{Oversampling}
	Sampling a signal at a rate higher than what the Nyquist-Shannon sampling theorem would dictate.
\end{block}
\begin{block}{Noise Shaping}
	The use of a filter to push quantization noise out of the signal band by wrapping the quantizer in a feedback loop.
\end{block}

\end{frame}

\begin{frame}{Basic Structure of a Sigma Delta Modulator}

\begin{figure}
\noindent\makebox[\textwidth]{
	\centering
	\begin{tikzpicture}[ampersand replacement=\&,scale=0.75, every node/.style={scale=0.75}]
		\matrix (m2) at (0,0) [row sep=0mm, column sep=5mm, matrix anchor=north west]
		{
			%-----------------------------------------------------------------------------------------------------------------------------------------------
			\node[coordinate]								(m2-00) {};							\&
			\node[coordinate]								(m2-01) {};							\&
			\node[coordinate,label={above:\PulseHigh \ $OSR \cdot f_s$}]	(m2-02) {};							\&
			\node[coordinate]								(m2-03) {};							\&
			\node[coordinate]								(m2-04) {};							\&
			\node[dspnodeopen]							(m2-05) {};							\& \\
			%------------------------------------------------------------------------------------------------------------------------------------------------
			\node[dspnodeopen,color=Red]						(m2-10) {};							\&
			\node[dspsquare,label={above:AAF}]					(m2-11) {};							\&
			\node[dspsquare]								(m2-12) {S/H};						\&
			\node[dspadder,label={below left:$-$}]				(m2-13) {};							\&
			\node[dspsquare,label={above:LF}]					(m2-14) {$\int$};						\&
			\node[dspsquare]								(m2-15) {\RaisingEdge};					\&
			\node[dspnodefull,color=ForestGreen]					(m2-16) {};							\&
			\node[dspsquare,label={above:DRF}]					(m2-17) {};							\&
			\node[dspfilter]								(m2-18) {$\downarrow OSR$};				\&
			\node[dspnodeopen,color=Blue]						(m2-19) {};							\& \\
			%------------------------------------------------------------------------------------------------------------------------------------------------
			\node[coordinate]								(m2-20) {};							\&
			\node[coordinate]								(m2-21) {};							\&
			\node[coordinate]								(m2-22) {};							\&
			\node[coordinate]								(m2-23) {};							\&
			\node[coordinate]								(m2-24) {};							\&
			\node[coordinate]								(m2-25) {};							\&
			\node[coordinate]								(m2-26) {};							\& \\
			%------------------------------------------------------------------------------------------------------------------------------------------------
		};
		
		\draw[->] (m2-02) -- (m2-12);
		\draw[dspconn] (m2-05) -- (m2-15);
		\draw[dspconn,color=Red] (m2-10) -- (m2-11);
		\draw[dspconn,color=Red] (m2-11) -- (m2-12);
		\draw[dspconn,color=ForestGreen] (m2-12) -- (m2-13);
		\draw[dspconn,color=ForestGreen] (m2-13) -- (m2-14);
		\draw[dspconn,color=ForestGreen] (m2-14) -- (m2-15);
		\draw[dspline,color=ForestGreen] (m2-15) -- (m2-16);
		\draw[dspconn,color=ForestGreen] (m2-16) -- (m2-17); 
		\draw[dspconn,color=ForestGreen] (m2-17) -- (m2-18);
		\draw[dspconn,color=Blue] (m2-18) -- (m2-19);
		\draw[dspline,color=ForestGreen] (m2-16) -- (m2-26);
		\draw[dspline,color=ForestGreen] (m2-26) -- (m2-23);
		\draw[dspconn,color=ForestGreen] (m2-23) -- (m2-13);
		
		\node[draw,inner xsep=15pt,inner ysep=10pt,dashed,fit={($(m2-05.north)+(-0.3,0)$) ($(m2-15.south)+(0.3,0.1)$)},label={[align=center]above:Linear Model}] {};
		
		\pgfplotsset{width=2.25cm,height=2.25cm,xmin=0.1,xmax=1000000,ymin=0.0001,ymax=10}
		\begin{loglogaxis}[
				at={($(m2-11) + (-9,-9)$)},
				ticks=none,
				axis x line*=bottom,
				axis y line*=left
			 ]
			\addplot[domain=1:100000]  {(60*x+10000)/(x*x + 60*x+10000)};
		\end{loglogaxis}
		\pgfplotsset{width=2.25cm,height=2.25cm,xmin=0.1,xmax=1000000,ymin=0.0001,ymax=10}
		\begin{loglogaxis}[
				at={($(m2-17) + (-9,-9)$)},
				ticks=none,
				axis x line*=bottom,
				axis y line*=left
			 ]
			\addplot[domain=1:100000]  {(60*x+10000)/(x*x + 60*x+10000)};
		\end{loglogaxis}

		\draw[Gray, ->, out=0, in=90, looseness=0.85] ($(m2-05)+(0.25, 0)$) to node[above,xshift=5pt] {NTF} ($(m2-16)+(0.4, 0.25)$);
		\draw[Gray, ->, out=-45, in=-90, looseness=0.5] ($(m2-12)+(1,-0.25)$) to node[below] {STF} ($(m2-16)+(0.4,-0.25)$);

	\end{tikzpicture}}
	\caption{A simplified block diagram of a discrete-time sigma delta A/D converter.}
\end{figure}
\vspace{-0.5cm}
\begin{figure}
\noindent\makebox[\textwidth]{
	\centering
	\begin{tikzpicture}[ampersand replacement=\&,scale=0.75, every node/.style={scale=0.75}]
		\matrix (m2) at (0,0) [row sep=0mm, column sep=5mm, matrix anchor=north west]
		{
			%-----------------------------------------------------------------------------------------------------------------------------------------------
			\node[coordinate]								(m2-00) {};							\&
			\node[coordinate]								(m2-01) {};							\&
			\node[coordinate]								(m2-02) {};							\&
			\node[coordinate]								(m2-03) {};							\&
			\node[coordinate,label={above:\PulseHigh \ $OSR \cdot f_s$}]	(m2-04) {};							\&
			\node[dspnodeopen]							(m2-05) {};							\& \\
			%------------------------------------------------------------------------------------------------------------------------------------------------
			\node[dspnodeopen,color=Red]						(m2-10) {};							\&
			\node[dspsquare,dashed,label={above:AAF}]				(m2-11) {};							\&
			\node[dspadder,label={below left:$-$}]				(m2-12) {};							\&
			\node[dspsquare,label={above:LF}]					(m2-13) {$\int$};						\&
			\node[dspsquare]								(m2-14) {S/H};						\&
			\node[dspsquare]								(m2-15) {\RaisingEdge};					\&
			\node[dspnodefull,color=ForestGreen]					(m2-16) {};							\&
			\node[dspsquare,label={above:DRF}]					(m2-17) {};							\&
			\node[dspfilter]								(m2-18) {$\downarrow OSR$};				\&
			\node[dspnodeopen,color=Blue]						(m2-19) {};							\& \\
			%------------------------------------------------------------------------------------------------------------------------------------------------
			\node[coordinate]								(m2-20) {};							\&
			\node[coordinate]								(m2-21) {};							\&
			\node[coordinate]								(m2-22) {};							\&
			\node[coordinate]								(m2-23) {};							\&
			\node[dspadc]								(m2-24) {D/A};							\&
			\node[coordinate]								(m2-25) {};							\&
			\node[coordinate]								(m2-26) {};							\& \\
			%------------------------------------------------------------------------------------------------------------------------------------------------
		};
		
		\draw[->] (m2-04) -- (m2-14);
		\draw[dspconn] (m2-05) -- (m2-15);
		\draw[dspconn,color=Red] (m2-10) -- (m2-11);
		\draw[dspconn,color=Red] (m2-11) -- (m2-12);
		\draw[dspconn,color=Red] (m2-12) -- (m2-13);
		\draw[dspconn,color=Red] (m2-13) -- (m2-14);
		\draw[dspconn,color=ForestGreen] (m2-14) -- (m2-15);
		\draw[dspline,color=ForestGreen] (m2-15) -- (m2-16);
		\draw[dspconn,color=ForestGreen] (m2-16) -- (m2-17); 
		\draw[dspconn,color=ForestGreen] (m2-17) -- (m2-18);
		\draw[dspconn,color=Blue] (m2-18) -- (m2-19);
		\draw[dspline,color=ForestGreen] (m2-16) -- (m2-26);
		\draw[dspconn,color=ForestGreen] (m2-26) -- (m2-24);
		\draw[dspline,color=Red] (m2-24) -- (m2-22);
		\draw[dspconn,color=Red] (m2-22) -- (m2-12);
		\pgfplotsset{width=2.25cm,height=2.25cm,xmin=0.1,xmax=1000000,ymin=0.0001,ymax=10}
		
		\node[draw,inner xsep=15pt,inner ysep=10pt,dashed,fit={($(m2-05.north)+(-0.3,0)$) ($(m2-15.south)+(0.3,0.1)$)},label={[align=center]above:Linear Model}] {};
		
		\begin{loglogaxis}[
				at={($(m2-11) + (-9,-9)$)},
				ticks=none,
				axis x line*=bottom,
				axis y line*=left
			 ]
			\addplot[domain=1:100000]  {(60*x+10000)/(x*x + 60*x+10000)};
		\end{loglogaxis}
		\pgfplotsset{width=2.25cm,height=2.25cm,xmin=0.1,xmax=1000000,ymin=0.0001,ymax=10}
		\begin{loglogaxis}[
				at={($(m2-17) + (-9,-9)$)},
				ticks=none,
				axis x line*=bottom,
				axis y line*=left
			 ]
			\addplot[domain=1:100000]  {(60*x+10000)/(x*x + 60*x+10000)};
		\end{loglogaxis}

		\draw[Gray, ->, out=0, in=90, looseness=0.85] ($(m2-05)+(0.25, 0)$) to node[above,xshift=5pt] {NTF} ($(m2-16)+(0.4, 0.25)$);
		\draw[Gray, ->, out=-75, in=-90, looseness=0.8] ($(m2-11)+(1,-0.25)$) to node[above] {STF} ($(m2-16)+(0.4,-0.25)$);

	\end{tikzpicture}}
	\caption{A simplified block diagram of a continuous-time sigma delta A/D converter.}
\end{figure}

\end{frame}

\begin{frame}{Design of the Loop Filter}
	The nonlinear quantizer in the forward path makes stability analysis difficult.
	Loop filter design is commonly done in one of several ways:
	\begin{itemize}
		\item Pure integrator -- DC-stable for low order loops.
		\item Prototype NTF -- noise rejection of linear model chosen from a family of filters.
		\item Optimization-based approaches -- wide range of techniques.
	\end{itemize}
	The design process often relies on extensive simulation to confirm that stability is likely during normal operation and the circuit may include complicated instability detection and recovery mechanisms.
\end{frame}


\begin{frame}[fragile]{Sections}
  Sections group slides of the same topic

  \begin{verbatim}    \section{Elements}\end{verbatim}

  for which \themename provides a nice progress indicator \ldots
\end{frame}

\section{Titleformats}

\begin{frame}{Metropolis titleformats}
    \themename supports 4 different titleformats:
    \begin{itemize}
        \item Regular
        \item \textsc{Smallcaps}
        \item \textsc{allsmallcaps}
        \item ALLCAPS
    \end{itemize}
    They can either be set at once for every title type or individually.
\end{frame}

{
    \metroset{titleformat frame=smallcaps}
\begin{frame}{Small caps}
    This frame uses the \texttt{smallcaps} titleformat.

    \begin{alertblock}{Potential Problems}
        Be aware, that not every font supports small caps. If for example you typeset your presentation with pdfTeX and the Computer Modern Sans Serif font, every text in smallcaps will be typeset with the Computer Modern Serif font instead.
    \end{alertblock}
\end{frame}
}

{
\metroset{titleformat frame=allsmallcaps}
\begin{frame}{All small caps}
    This frame uses the \texttt{allsmallcaps} titleformat.

    \begin{alertblock}{Potential problems}
        As this titleformat also uses smallcaps you face the same problems as with the \texttt{smallcaps} titleformat. Additionally this format can cause some other problems. Please refer to the documentation if you consider using it.

        As a rule of thumb: Just use it for plaintext-only titles.
    \end{alertblock}
\end{frame}
}

{
\metroset{titleformat frame=allcaps}
\begin{frame}{All caps}
    This frame uses the \texttt{allcaps} titleformat.

    \begin{alertblock}{Potential Problems}
        This titleformat is not as problematic as the \texttt{allsmallcaps} format, but basically suffers from the same deficiencies. So please have a look at the documentation if you want to use it.
    \end{alertblock}
\end{frame}
}

\section{Elements}

\begin{frame}[fragile]{Typography}
      \begin{verbatim}The theme provides sensible defaults to
\emph{emphasize} text, \alert{accent} parts
or show \textbf{bold} results.\end{verbatim}

  \begin{center}becomes\end{center}

  The theme provides sensible defaults to \emph{emphasize} text,
  \alert{accent} parts or show \textbf{bold} results.
\end{frame}

\begin{frame}{Font feature test}
  \begin{itemize}
    \item Regular
    \item \textit{Italic}
    \item \textsc{SmallCaps}
    \item \textbf{Bold}
    \item \textbf{\textit{Bold Italic}}
    \item \textbf{\textsc{Bold SmallCaps}}
    \item \texttt{Monospace}
    \item \texttt{\textit{Monospace Italic}}
    \item \texttt{\textbf{Monospace Bold}}
    \item \texttt{\textbf{\textit{Monospace Bold Italic}}}
  \end{itemize}
\end{frame}

\begin{frame}{Lists}
  \begin{columns}[T,onlytextwidth]
    \column{0.33\textwidth}
      Items
      \begin{itemize}
        \item Milk \item Eggs \item Potatos
      \end{itemize}

    \column{0.33\textwidth}
      Enumerations
      \begin{enumerate}
        \item First, \item Second and \item Last.
      \end{enumerate}

    \column{0.33\textwidth}
      Descriptions
      \begin{description}
        \item[PowerPoint] Meeh. \item[Beamer] Yeeeha.
      \end{description}
  \end{columns}
\end{frame}
\begin{frame}{Animation}
  \begin{itemize}[<+- | alert@+>]
    \item \alert<4>{This is\only<4>{ really} important}
    \item Now this
    \item And now this
  \end{itemize}
\end{frame}
\begin{frame}{Figures}
  \begin{figure}
    \newcounter{density}
    \setcounter{density}{20}
    \begin{tikzpicture}
      \def\couleur{alerted text.fg}
      \path[coordinate] (0,0)  coordinate(A)
                  ++( 90:5cm) coordinate(B)
                  ++(0:5cm) coordinate(C)
                  ++(-90:5cm) coordinate(D);
      \draw[fill=\couleur!\thedensity] (A) -- (B) -- (C) --(D) -- cycle;
      \foreach \x in {1,...,40}{%
          \pgfmathsetcounter{density}{\thedensity+20}
          \setcounter{density}{\thedensity}
          \path[coordinate] coordinate(X) at (A){};
          \path[coordinate] (A) -- (B) coordinate[pos=.10](A)
                              -- (C) coordinate[pos=.10](B)
                              -- (D) coordinate[pos=.10](C)
                              -- (X) coordinate[pos=.10](D);
          \draw[fill=\couleur!\thedensity] (A)--(B)--(C)-- (D) -- cycle;
      }
    \end{tikzpicture}
    \caption{Rotated square from
    \href{http://www.texample.net/tikz/examples/rotated-polygons/}{texample.net}.}
  \end{figure}
\end{frame}
\begin{frame}{Tables}
  \begin{table}
    \caption{Largest cities in the world (source: Wikipedia)}
    \begin{tabular}{lr}
      \toprule
      City & Population\\
      \midrule
      Mexico City & 20,116,842\\
      Shanghai & 19,210,000\\
      Peking & 15,796,450\\
      Istanbul & 14,160,467\\
      \bottomrule
    \end{tabular}
  \end{table}
\end{frame}
\begin{frame}{Blocks}
  Three different block environments are pre-defined and may be styled with an
  optional background color.

  \begin{columns}[T,onlytextwidth]
    \column{0.5\textwidth}
      \begin{block}{Default}
        Block content.
      \end{block}

      \begin{alertblock}{Alert}
        Block content.
      \end{alertblock}

      \begin{exampleblock}{Example}
        Block content.
      \end{exampleblock}

    \column{0.5\textwidth}

      \metroset{block=fill}

      \begin{block}{Default}
        Block content.
      \end{block}

      \begin{alertblock}{Alert}
        Block content.
      \end{alertblock}

      \begin{exampleblock}{Example}
        Block content.
      \end{exampleblock}

  \end{columns}
\end{frame}
\begin{frame}{Math}
  \begin{equation*}
    e = \lim_{n\to \infty} \left(1 + \frac{1}{n}\right)^n
  \end{equation*}
\end{frame}
\begin{frame}{Line plots}
  \begin{figure}
    \begin{tikzpicture}
      \begin{axis}[
        mlineplot,
        width=0.9\textwidth,
        height=6cm,
      ]

        \addplot {sin(deg(x))};
        \addplot+[samples=100] {sin(deg(2*x))};

      \end{axis}
    \end{tikzpicture}
  \end{figure}
\end{frame}
\begin{frame}{Bar charts}
  \begin{figure}
    \begin{tikzpicture}
      \begin{axis}[
        mbarplot,
        xlabel={Foo},
        ylabel={Bar},
        width=0.9\textwidth,
        height=6cm,
      ]

      \addplot plot coordinates {(1, 20) (2, 25) (3, 22.4) (4, 12.4)};
      \addplot plot coordinates {(1, 18) (2, 24) (3, 23.5) (4, 13.2)};
      \addplot plot coordinates {(1, 10) (2, 19) (3, 25) (4, 15.2)};

      \legend{lorem, ipsum, dolor}

      \end{axis}
    \end{tikzpicture}
  \end{figure}
\end{frame}
\begin{frame}{Quotes}
  \begin{quote}
    Veni, Vidi, Vici
  \end{quote}
\end{frame}

{%
\setbeamertemplate{frame footer}{My custom footer}
\begin{frame}[fragile]{Frame footer}
    \themename defines a custom beamer template to add a text to the footer. It can be set via
    \begin{verbatim}\setbeamertemplate{frame footer}{My custom footer}\end{verbatim}
\end{frame}
}

\begin{frame}{References}
  Some references to showcase [allowframebreaks] \cite{knuth92,ConcreteMath,Simpson,Er01,greenwade93}
\end{frame}

\section{Conclusion}

\begin{frame}{Summary}

  You can view the source on Github
  
  \begin{center}\url{github.com/JoeyEremondi/UBC-Metropolis-Beamer}\end{center}

  The original theme can be found at

  \begin{center}\url{github.com/matze/mtheme}\end{center}

  The theme \emph{itself} is licensed under a
  \href{http://creativecommons.org/licenses/by-sa/4.0/}{Creative Commons
  Attribution-ShareAlike 4.0 International License}.

  \begin{center}\ccbysa\end{center}

\end{frame}

{\setbeamercolor{palette primary}{fg=black, bg=yellow}
\begin{frame}[standout]
  Questions?
\end{frame}
}

\appendix

\begin{frame}[fragile]{Backup slides}
  Sometimes, it is useful to add slides at the end of your presentation to
  refer to during audience questions.

  The best way to do this is to include the \verb|appendixnumberbeamer|
  package in your preamble and call \verb|\appendix| before your backup slides.

  \themename will automatically turn off slide numbering and progress bars for
  slides in the appendix.
\end{frame}

\begin{frame}[allowframebreaks]{References}

  \bibliography{demo}
  \bibliographystyle{abbrv}

\end{frame}

\end{document}

